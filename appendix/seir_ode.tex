% !TeX root = ../main.tex


\begin{tcolorbox}[title=決定性 SEIR 模型 (Deterministic SEIR)]

決定性 SEIR 模型使用的是微分方程系統(Ordinary Differential Equations),將各族群在 $t$ 時間點的改變傾向用方程式表達出來。在模型中 (Fig. \ref{fig:seir-ode}),分成未接觸病毒的人($S$, Eq. \ref{eq:ode-s}), 暴露病毒的人($E$, Eq. \ref{eq:ode-e}), 有感染力的病人 ($I_1$, Eq. \ref{eq:ode-i1}), 無症狀感染者 ($I_2$, Eq. \ref{eq:ode-i2}) 以及 康復的人($R$, Eq. \ref{eq:ode-r})


\begin{align} 
    \frac{dS}{dt} &= -\beta(t)\frac{S(I_1 + I_2)^{\alpha}}{N} \label{eq:ode-s}\\
    \frac{dE}{dt} &= \beta(t)\frac{S(I_1 + I_2)^{\alpha}}{N} - \sigma E \label{eq:ode-e}\\
    \frac{dI_1}{dt} &= \sigma E - \gamma_1 I_1 \label{eq:ode-i1}\\
    \frac{dI_2}{dt} &= \gamma_1 I_1 - \gamma_2 I_2 \label{eq:ode-i2}\\
    \frac{dR}{dt} &= \gamma_2 I_2 \label{eq:ode-r} 
\end{align}

$\alpha$ 指的是族群混合的程度係數; $\sigma$ 是暴露病毒的人($E$)變成有感染力($I_1$) 的比率; $\gamma_1$ 是有感染力的人($I_1$) 變成無症狀感染($I_2$) 的比率; $\gamma_2$ 是康復的比率。

\end{tcolorbox}
